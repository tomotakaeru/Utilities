\usepackage{amsmath,amssymb} %数式
\usepackage{bm} %ベクトル表記
\usepackage{mathtools} %行列要素を右寄せにできる
\usepackage[dvipdfmx]{graphicx} %figure
\usepackage[dvipdfmx]{color} %文字色
\usepackage{listings,plistings} %ソースコード貼り付け
% \usepackage{url} %URLリンク貼り付け
\usepackage[dvipdfmx]{hyperref} %ハイパーリンク
\usepackage{siunitx} %SI単位系
\usepackage{fancyhdr} %ヘッダ,フッタ
\usepackage[left=19.05mm, right=19.05mm, top=25.40mm, bottom=25.40mm]{geometry}
\usepackage{multicol} %一部段組
\usepackage{multirow} %表セル縦結合
\usepackage{lscape} %表を90°回転
\usepackage{here} %図を強制的に指定位置に置く
%
\newcommand{\diver}{\mathrm{div}\,}
\newcommand{\grad}{\mathrm{grad}\,}
\newcommand{\rot}{\mathrm{rot}\,}
\newcommand{\bhline}[1]{\noalign{\hrule height #1}} %表の横罫線の太さ変更\bhline{1.2pt}
\newlength\savedwidth
\newcommand{\bcline}[1]{\noalign{\global\savedwidth\arrayrulewidth\global\arrayrulewidth 1.2pt} \cline{#1}
\noalign{\global\arrayrulewidth\savedwidth}} %表の横罫線の太さ変更\bcline{1-3} %1.2pt固定
\newcommand{\bvline}[1]{\vrule width #1} %表の縦罫線の太さ変更
%
\renewcommand{\contentsname}{Contents} %目次名変更
% \renewcommand{\thesubsection}{\arabic{subsection}.} %subsectionの表記変更
% \renewcommand{\thesubsubsection}{\arabic{subsection}.\arabic{subsubsection}} %subsubsectionの表記変更
\renewcommand{\lstlistingname}{Src.} %lstlistingの表記変更
\renewcommand{\figurename}{Fig.} %figureの表記変更
\renewcommand{\tablename}{Table.} %tableの表記変更
\renewcommand{\labelitemi}{$\triangleright$} %itemの表記変更
% \renewcommand{\labelenumi}{\textcircled{\scriptsize \arabic{enumi}}} %enumerateの表記変更
\renewcommand{\refname}{References} %参考文献の表記変更
\renewcommand{\abstractname}{Abstract} %概要の表記変更
\renewcommand{\arraystretch}{1.1} %表の高さ変更
%
\def\seireki{\the\year/\ifnum\month<10 0\fi\the\month/\ifnum\day<10 0\fi\the\day} %日付を西暦表示 20xx/yy/zz
%
% \makeatletter %数式番号にsection番号追加
% \@addtoreset{equation}{section}
% \def\theequation{\thesection.\arabic{equation}}
% \makeatother
%
\pagestyle{fancy} %ヘッダ,フッタ
\renewcommand{\headrulewidth}{0.4pt} %ヘッダ罫線
\renewcommand{\footrulewidth}{0pt} %フッタ罫線
\lhead{中須賀・船瀬研究室 輪講資料}
% \rhead{\seireki}
\rhead{2019/11/19}
% \let\origtitle\title %タイトルを\footに
% \renewcommand{\title}[1]{\lfoot{#1}\origtitle{#1}}
% \lfoot{\leftmark \quad \rightmark}
\cfoot{\thepage}
% \rfoot{}
%
\lstset{ %ソースコードの見た目設定
  basicstyle={\ttfamily},
  commentstyle={\small\color[rgb]{0.6,0.1,0.1}},
  keywordstyle={\bfseries\color[rgb]{0.4,0.8,0.4}},
  ndkeywordstyle={\small},
  stringstyle={\small\ttfamily},
  identifierstyle={\ttfamily\bfseries\color[rgb]{0.3,0.3,0.8}}, 
  numberstyle={\scriptsize},
  frame={tb},
  numbers=left,
  stepnumber=1,
  numbersep=1zw,
  xleftmargin=3zw,
  xrightmargin=0zw,
  lineskip=-0.5ex,
  columns=[l]{fullflexible},
  breaklines=true,
  keepspaces=true,
  showstringspaces=false
}
%
%\maketitleの変更(seminar)
\makeatletter
\def\@maketitle{
\begin{center}
\vspace*{5truept}
% \vspace*{10truept}
{\LARGE\bf \@title \par} % タイトル
\vspace{5mm}
% {\large \@date \par} % 提出年月日
% \vspace{2mm}
% {\@department \par} % 所属
{\large \@author \par} % 氏名
\par\vskip 2.5em
\end{center}
}
\makeatother
